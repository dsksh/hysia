\documentclass[12pt,a4paper]{article}

\usepackage[utf8]{inputenc}
\usepackage[english]{babel}
%\usepackage[english]{isodate}
%\usepackage[parfill]{parskip}

\usepackage{graphicx}

\usepackage{amsmath}
\usepackage{amssymb}
\usepackage{newtxtext,newtxmath}
\usepackage{bm}
\usepackage{listings}
\usepackage{color}
\usepackage{url}
\usepackage{intmacros}
\usepackage{syntax}
\usepackage{manual}

\lstset{%
%    backgroundcolor=\color{yellow!20},%
    basicstyle=\ttfamily,%
    %numbers=left, numberstyle=\tiny, stepnumber=2, numbersep=5pt,%
    numbers=left, numberstyle=\tiny, numbersep=5pt,%
	frame=single,%
    }%

\lstset{emph={%  
    let, var, init, at, wait, once, when, goto, then, end, prop, param%
    },emphstyle={\color{blue}\bfseries}%
}

\newcommand{\LNCS}[1]{\textit{LNCS}~#1}
\newcommand{\ENTCS}[1]{\textit{ENTCS}~#1}

\begin{document}

\title{HySIA Manual (version~0.1.1)}
\author{Daisuke Ishii \quad \texttt{dsksh@acm.org}}

\maketitle

\setlength{\grammarparsep}{8pt plus 1pt minus 1pt} % increase separation between rules
\setlength{\grammarindent}{8em} % increase separation between LHS/RHS 

\emph{HySIA} is a reliable simulator and verifier for hybrid systems.
%
HySIA supports nonlinear \emph{hybrid automata} (HAs) whose ODEs, guards, and reset functions are specified with nonlinear expressions. It assumes a deterministic class of HA; a transition to another location happens whenever a guard condition holds.
%
Main functionalities of HySIA are as follows:

\textbf{Simulation}.
%
HySIA simulates an HA based on interval analysis;
it computes an overapproximation of a bounded trajectory that is composed of \emph{boxes} (i.e., closed interval vectors) and \emph{parallelotopes} (linear transformation of boxes).
%
Intensive use of interval analysis techniques distinguishes HySIA from other reachability analysis tools.
%
First, the simulation process carefully reduces \emph{wrapping effect} that can expand an enclosure interval. As a result, HySIA is able to simulate HA for more number of steps than other overapproximation-based tools; e.g., it can simulate a periodic bouncing ball for more than a thousand steps.
%
Second, HySIA assures the soundness of each interval computation, so that the resulting overapproximation is verified to contain a theoretical trajectory.
%This verification often fails, e.g., when an ODE is \emph{stiff} or when a trajectory and a guard are close to tangent.
%Thanks to this \emph{quasi-complete} manner, the simulation process of HySIA is decidable and performs efficiently.
%
See Reference~\cite{GI2016} for details of the underlying method.

\textbf{Monitoring}.
%
HySIA takes a temporal property as an input and monitors whether a simulated trajectory satisfies the property; otherwise, HySIA computes (an interval overapproximation of) a robustness signal for the property.
The verification process is based on a monitoring procedure of the \emph{signal temporal logic} (STL) formulas~\cite{Maler2004}, which is extended to handle overapproximation of trajectories.
%The soundness of interval computation is again utilized here to evaluate the logical negation against an approximated trajectory.
%
See Reference~\cite{Ishii2016IEICE} for more details.


\section{Short Tutorial}

HySIA provides a specification language for hybrid automata and temporal logic properties.
An HA is simulated and analyzed about a property using the \verb|hysia| command.
This section exemplifies the modeling, simulation, and verification of an HA using HySIA.

\subsection{Modeling Hybrid Automata}

Below is an example specification that describes a simple bouncing ball model (\texttt{bb.ha}):
\begin{lstlisting}
(* first part *)
let   g = 1
let   c = 0.9
let   pertb = [-1e-5, 1e-5]

(* second part *)
var   y, vy

init  Loc, 1+pertb, 0+pertb

at Loc wait vy, -g
      once (y, -vy) goto Loc then y, -c*vy
end

(* third part *)
param order = 20
param t_max = 100
param dump_interval = 0.1
\end{lstlisting}
%
The specification consists of three parts.
The first part (Lines~1--4) defines three constants \verb|g|, \verb|c|, and \verb|pertb| to be used in the second part.
The values of constants can be either real values or intervals.

The second part (Lines~6--13) describes an HA.
Line~7 declared the state variables \verb|y| and \verb|vy| of the HA, which are evaluated over timeline.
Line~9 describes the initial state as a comma-separated list of a location and values for each state variable.
Lines~11--13 defines a location named \verb|Loc|.
After the keyward \verb|wait|, the derivatives of the state variable are specified as $\frac{d}{dt}\mathtt{y} = \mathtt{vy}$ and $\frac{d}{dt}\mathtt{vy} = -\mathtt{g}$.
At Line~12, an inter-location transition is specified.
The tuple after \verb|once| tells that the guard condition for this transition is $\mathtt{y} = 0 ~\land~ \mathtt{vy} > 0$. (Only the left-hand side of the implicit form is described.)
Whenever the guard is satisfied, an execution will transit to the same location \verb|Loc| with a reset of the state variables as $\mathtt{y} := \mathtt{y}$ and $\mathtt{vy} := -\mathtt{c}\,\mathtt{vy}$.

The last part (Lines~15--18) configures some parameters of the simulator implementation, i.e., the order \verb|order| of Taylor coefficient expansion, the time horizon \verb|t_max| of a continuous state evolution, and \verb|dump_interval| that bounds the step size when computing a dumped data.

In general, an HA consists of multiple locations. The bouncing ball system can be modeled with two locations by modelling the second part as follows: %(\texttt{bb1.ha}):
\begin{lstlisting}
init  Fall, 1+pertb, -0+pertb

at Fall wait vy, -g
      once (y, vy) goto Rise then y, -c*vy
end

at Rise wait vy, -g
      once (vy, -y) goto Fall then y, vy
end
\end{lstlisting}

\subsection{Simulation of a Model}

Once a specification is prepared, a user can simulate the model for 10 transitions with the following command:
\begin{verbatim}
$ hysia bb.ha -n 10 -dump
\end{verbatim}
The option \verb|-dump| let HySIA to output the result of the simulation to the file \verb|pped.dat| in a JSON format.
The output data consists of a set of \emph{boxes} (interval vectors) that encloses the trajectories of the HA.
(In this example, the interval value \verb|pertb| allows perturbation of trajectories; these trajectories are enclosed within the boxes.)
The result can be visualized as shown in Figure~\ref{f:bb}.

\begin{figure}[t]
\centering
\includegraphics[width=0.8\linewidth]{bb.eps}
\caption{Dumped trajectory enclosure of the bouncing ball example}
\label{f:bb}
\end{figure}

Simulation and verification of HySIA are all computed with validated interval analysis.
Thus, a resulting interval enclosure of a system's state expands as longer the simulation length and more the \emph{wrapping effect} occurs.
However, thanks to the underlying \emph{parallelotope}-based simulation method for wrapping effect reduction, HySIA is able to simulate a large number of jumps for various HA.
Indeed, for the bouncing ball model, when we modify the parameter values as \verb|c = 1| and \verb|pertb = 0|, HySIA can simulate for more than a thousand steps.

As we have seen so far, HySIA allows models to involve some uncertainties derived by interval values.
HySIA computes (the overapproximation of) the reachable region with respect to the uncertainties.
However, here we may encounter a drawback of HySIA; a simulation is not always successful but may result in an error.
%
When we modify the value of \verb|pertb| to a slightly large interval \verb|[-1e-4, 1e-4]| and run a simulation, \verb|hysia| will output as follows:
\begin{verbatim}
$ hysia bb.ha -n 10 -dump
step 0 (0.000000 < inf) at Loc
step 1 (1.414043 < inf) at Loc
step 2 (3.959155 < inf) at Loc
step 3 (6.249714 < inf) at Loc
step 4 (8.310768 < inf) at Loc
step 5 (10.163552 < inf) at Loc
step 6 (11.809965 < inf) at Loc
libc++abi.dylib: terminating with uncaught exception of type std::
runtime_error: zero in the derivative
Abort trap: 6
\end{verbatim}
The simulation fails after the sixth jump. The output implies that HySIA fails in the detection of a discrete change because a state enclosure becomes too large.
In fact, in each detection, HySIA checks whether the orientation between the trajectory and the guard is regular enough so that it is sure that each trajectory within an enclosure satisfies the guard.


\subsection{Verification of STL Properties}

HySIA provides a function for verifying properties described in the \emph{signal temporal logic} (STL).
For example, we can add an STL property 
\begin{lstlisting}
prop G[0,10] F[0,1] y-0.3
\end{lstlisting}
in the above mentioned model.
(It should be added in between the second and third parts.)
The property intuitively claims that, for the duration of $10$ time units, the height of the ball (\verb|y|) goes beyond $0.3$ (\verb|y-0.3| is interpreted as $\texttt{y} - 0.3 > 0$) within every duration of $1$ time unit.

The verification can be done with the following command:
\begin{verbatim}
$ hysia bb.ha -a
\end{verbatim}
The \verb|hysia| command calculates the \emph{necessary} simulation length for the verification (in this case, 11 time units), performs a simulation, and evaluates the satisfiability of this property.
Accordingly, the verification succeeds with the following output:
\begin{verbatim}
$ hysia bb.ha -a
step 0 (0.000000 < 11.000000) at Loc
step 1 (1.414196 < 11.000000) at Loc
step 2 (3.959734 < 11.000000) at Loc
step 3 (6.250722 < 11.000000) at Loc
step 4 (8.312608 < 11.000000) at Loc
step 5 (10.168295 < 11.000000) at Loc
true, 0.036239
\end{verbatim}


\section{Getting the Tool}

The HySIA tool can be obtained and used in three ways.

\subsection{Web Demonstration}

A web demonstration site is available at:
\begin{quote}
\url{http://bit.ly/hysia}
\end{quote}
%
Through a web browser, you can access a GUI, load the basic examples, modify the HA/STL specification, simulate, and verify the specification.


\subsection{Docker Image}

For those who are familier with \emph{Docker},\footnote{\url{https://www.docker.com/}} Docker containers are available at: 
\begin{quote}        
\url{https://hub.docker.com/r/dsksh/}
\end{quote}        

The image \verb|dsksh/hysia-web| contains the server program for the web demonstration. A container can be launched by:
\begin{verbatim}
$ docker run -p 8080:80 dsksh/hysia-web
\end{verbatim}
Then, the server will be available at:
\begin{quote}
\url{http://localhost:8080}
\end{quote}


\subsection{Source Distribution}

The HySIA source code is distributed via GitHub:
\begin{quote}
\url{https://github.com/dsksh/hysia}.
\end{quote}

\subsubsection*{Requirements}

In addition to a standard UNIX-like environment, the following softwares are required to compile HySIA:
\begin{itemize}
    \item \emph{C/C++ compiler}.
        We have compiled with both \verb|gcc| (versions~4.7.4–4.8.4) and \verb|clang| (Apple LLVM version~8).
        %
    \item \emph{OCaml compiler}.
        We have tested with versions~3.12–4.02.
        %
    \item \emph{CAPD library}.\footnote{http://capd.ii.uj.edu.pl/}
        HySIA is built on an old release of the CAPD-DynSys 3.0 library, which was distributed around 2014, and is not available on the official site.
        A (slightly modified) source package is available at:
        \begin{quote}
        \url{https://www.dropbox.com/s/3uf7t2nsizfebno/capdDynSys-201406.zip}
        \end{quote}
        %
    \item \emph{Boost library}.\footnote{http://www.boost.org/}
        \verb|shared_ptr| is used.
        %
    \item \emph{OUnit}.\footnote{http://ounit.forge.ocamlcore.org/} Optional for test cases compilation.
    \item \emph{Eliom}.\footnote{http://ocsigen.org/eliom/} Optional for building the web application.
\end{itemize}


\subsubsection*{Build}

In the root directory of the source code, HySIA can be compiled with:
\begin{verbatim}
$ make
\end{verbatim}

Some modification, e.g., specification of compiler options and library pathes, might be needed in \verb|Makefile|.

When compilation succeeds, the program file
\begin{verbatim}
src_ocaml/hss.opt
\end{verbatim}
is generated.


\section{Examples}

\subsection{Simple Rotation System}

TBD.

\subsection{Bouncing Planet}

TBD.


\section{Reference Manual}

\subsection{Command-Line Tool}

The basic syntax for the command-line execution is as follows:
\begin{grammar}
<command> ::= `hysia' <options> <filename>
\end{grammar}
%
The \texttt{hysia} command accepts the following options:
\begin{description}
\item[\texttt{-h}, \texttt{-help}, or \texttt{--help}] Displays a summary of the options accepted by the command.
\item[\texttt{-n}] Specifies the number steps to simulate (default is $\infty$).
\item[\texttt{-t}] Specifies the max simulation time (default is $\infty$).
\item[\texttt{-a}] Decides the simulation length automatically from the STL property.
\item[\texttt{-g}] Sets the debug flag.
\item[\texttt{-dump}] Activates dumping plot to the file ``pped.dat.''
\item[\texttt{-cm_thres}] Sets the threshold for character matrix selection.
\end{description}

The option \texttt{-cm_thres} specifies the character matrix $B$ to be used in the parallelotope method (see the corresponding publication for the detail). It is selected as follows:
\begin{itemize}
	\item $-1$: $B := (\mathrm{mid}\J) A$.
	\item $0$: $B := I$ (i.e., identity matrix).
	\item $1$: $B := \mathrm{orthogonalize}((\mathrm{mid}\J) A)$.
	\item $n > 1$ (default): 
		\[
			B := \begin{cases}
				(\mathrm{mid}\J) A & \text{if $\kappa((\mathrm{mid}\J) A) < n$} \\
				\mathrm{orthogonalize}((\mathrm{mid}\J) A) & \text{otherwise.}
			\end{cases}
		\]
\end{itemize}


\subsection{Solving Parameters}

\begin{description}
\item[\texttt{order}] Order of Taylor expansion.
\item[\texttt{t_max}] Max time horizon assumed in the simulation of each step.
\item[\texttt{h_min}] Min time CAPD integration can take.
\item[\texttt{epsilon}] Specifies the precision of the event detection.
\item[\texttt{dump_interval}] Sets the precision of the dumped flowpipe data.
\item[\texttt{delta}] Parameter for the box inflation process.
\item[\texttt{tau}] Parameter for the box inflation process.
\item[\texttt{cm_thres}] Parameter for the character matrix selection.
\end{description}


\subsection{Specification Language}

This section describes the grammar of the specification language of HySIA.
A specification consists of the definition of a hybrid automaton, an STL formula, and solving parameter configurations.

\subsubsection{Lexical conventions}

The lexical class of digits, letters, and identifiers is the following:
\begin{grammar}
<digit> ::= [`0'--`9']

<letter> ::= [`a'--`z' `A'--`Z']

<id> ::= <letter> (<digit> | <letter> | `_')*
\end{grammar}

The syntax of various numeral expressions is as follows:
\begin{grammar}
<integer> ::= <digit>+

<float> ::= 
%<digit>+ ~~|~~ <digit>+ `.' <digit>*
<digit>+ ~ (`.' <digit>*)? ~ ( (`e'|`E') (`+'|`-')? <digit>+ )?
%\alt <digit>+ `.' <digit>* (`e'|`E')  <digit>+
%\alt <digit>+ (`e'|`E') <digit>+
%\alt <digit>+ (`e'|`E') (`+'|`-') <digit>+

<float-pn> ::= <float> ~~|~~ `-' <float>

<interval> ::= <interval-noun> ~|~ <float-pn>

<interval-noun> ::= `(' <float-pn> `,' <float-pn> `)'

<interval-list> ::= <interval> <interval-list-rest>
\alt `(' <interval> <interval-list-rest> `)'

<interval-list-rest> ::= `,' <interval> <interval-list-rest> ~~|~~ <empty>
\end{grammar}


\noindent
\textbf{Comments.}
Comments are either enclosed between \texttt{(*} and \texttt{*)} (can be nested) or prefixed with \texttt{\#}.


\subsubsection{Toplevel syntax}

The syntax for the toplevel of specifications is the following:
\begin{grammar}
<specification> ::= <statement>+ <property>? <solver-param>*

<statement> ::= 
`let' <id> `=' <interval>
\alt `let' <id> `=' `R' <float>
\alt `var' <var-list>
\alt `init' <expr-list>
\alt `at' <id> <flow> <invariant>? <edge>* `end'

<property> ::= `prop' <stl-formula>

<solver-param> ::= `param' <id> `=' <float-pn>


<flow> ::= `wait' <expr-list>

<invariant> ::= `inv' <expr-list>

<edge> ::= (`when' | `once') `(' <expr> `,' <expr-list> `)' `goto' <id> `then' <expr-list>
\end{grammar}


\subsubsection{Expressions}

The operators and function application in expressions have the priorities and associativities as shown in the table below (from lowest to greatest priority):
\begin{table}[ht]
	\centering
	%\caption{\label{t:operators} Terms} 
	%\small
    \begin{tabular}{|l|l|} \hline
		construct & associativity \\
		\hline
		`\texttt{+}', `\texttt{-}' & left \\
		`\texttt{*}', `\texttt{/}' & left \\
		function application & left \\
		`\texttt{\^}', `\texttt{-}' (unary) & --- \\
		\hline
	\end{tabular}
\end{table}

The syntax for expressions is the following:
\begin{grammar}
<expr> ::= <expr> `+' <expr> ~~|~~ <expr> `-' <expr>
\alt <expr> `*' <expr> ~~|~~ <expr> `/' <expr>
\alt `-' <expr> ~~|~~ <expr> `^' <integer>
\alt <function> <expr>
\alt <id> ~~|~~ <interval>
\alt `(' <expr> `)' 

<function> ::= `sqrt' | `exp' | `log' | `sin' | `cos' | `atan' | `asin' | `acos'

<expr-list> ::= <expr> <expr-list-rest> ~~|~~ `(' <expr> <expr-list-rest> `)'

<expr-list-rest> ::= `,' <expr> <expr-list-rest> ~|~ <empty>
\end{grammar}


\subsubsection{STL formulae}

The operators in formulae have the following priorities and associativities:
\begin{table}[ht]
	\centering
	%\caption{\label{t:operators} Terms} 
	%\small
    \begin{tabular}{|l|l|} \hline
		construct & associativity \\
		\hline
		`\texttt{->}' & right \\
		`\texttt{|}' & right \\
		`\texttt{\&}' & right \\
		`\texttt{U} [$\cdot$,$\cdot$]' & right \\
		`\texttt{G} [$\cdot$,$\cdot$]', `\texttt{F} [$\cdot$,$\cdot$]' & --- \\
		`\texttt{!}' & --- \\
		\hline
	\end{tabular}
\end{table}

The syntax for formulae is the following:
\begin{grammar}
<stl-formula> ::= `true' ~~|~~ `false'
\alt <id> ~~|~~ <expr>
\alt `!' <stl-formula>
\alt <stl-formula> (`&'|`|'|`->') <stl-formula>
\alt (`F'|`G') <noun-interval> <stl-formula>
\alt <stl-formula> `U' <noun-interval> <stl-formula>
\alt `(' <stl-formula> `)'
\end{grammar}


\begin{thebibliography}{99}
\bibitem{Donze2013}
    A. Donz\'{e}, T. Ferr\`{e}re, O. Maler: 
    {Efficient Robust Monitoring for STL},
    Proc. of CAV, pp.~264--279, \LNCS{8044}, 2013.

\bibitem{GI2016}
    A. Goldsztejn, D. Ishii: 
    {A Parallelotope Method for Hybrid System Simulation},
    Reliable Computing, 23:163--185, 2016.

\bibitem{Ishii2016IEICE}
    D. Ishii, N. Yonezaki, A. Goldsztejn: 
    {Monitoring Temporal Properties using Interval Analysis}. 
    IEICE Transactions on Fundamentals of Electronics, Communications and Computer Sciences,
    E99-A, 2016.

\bibitem{Ishii2015-NSV}
    D. Ishii, N. Yonezaki, A. Goldsztejn: 
    {Monitoring Bounded LTL Properties Using Interval Analysis}. 
    Proc. of 8th International Workshop on Numerical Software Verification (NSV), 
    \ENTCS{317}, pp.~85--100, 2015.

\bibitem{Maler2004}
    O. Maler and D. Nickovic:
    Monitoring Temporal Properties of Continuous Signals,
    Proc. of FORMATS, pp.~152--166, \LNCS{3253}, 2004.
\end{thebibliography}

\end{document}
