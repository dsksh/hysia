\documentclass[10pt,a4paper]{article}

\usepackage[utf8]{inputenc}
\usepackage[english]{babel}
%\usepackage[english]{isodate}
%\usepackage[parfill]{parskip}

\usepackage{amsmath}
\usepackage{amssymb}
\usepackage{bm}
\usepackage{listings}
\usepackage{color}
\usepackage{url}
\usepackage{intmacros}
\usepackage{syntax}
\usepackage{manual}

\lstset{%
%    backgroundcolor=\color{yellow!20},%
    basicstyle=\ttfamily,%
    %numbers=left, numberstyle=\tiny, stepnumber=2, numbersep=5pt,%
    numbers=left, numberstyle=\tiny, numbersep=5pt,%
	frame=single,%
    }%

\lstset{emph={%  
    let, var, init, at, wait, once, when, goto, then, end, param%
    },emphstyle={\color{blue}\bfseries}%
}

\begin{document}

\title{HySIA Manual (beta version)}
\author{Daisuke Ishii}

\maketitle

\setlength{\grammarparsep}{8pt plus 1pt minus 1pt} % increase separation between rules
\setlength{\grammarindent}{8em} % increase separation between LHS/RHS 

\section{Short Tutorial}

Specification of a simple bouncing ball model (\texttt{bb.ha}):
\begin{lstlisting}
let g = 1
let c = 1

let pertb = [-0.001, 0.001]

var   y, vy

init  Loc, 1+pertb, -0+pertb

at Loc wait vy, -g
  once (y, vy) goto Loc then y, -c*vy
end

param order = 20
param t_max = 100
param dump_interval = 0.1
\end{lstlisting}

A model with two locations (\texttt{bb1.ha}):
\begin{lstlisting}
(* ... *)

init  Fall, 1+pertb, -0+pertb

at Fall wait vy, -g
  once (y, vy) goto Rise then y, -c*vy
end

at Rise wait vy, -g
  once (vy, -y) goto Fall then y, vy
end

(* ... *)
\end{lstlisting}

Command-line execution for simulating for 10 steps:
\begin{verbatim}
$ hysia bb.ha -n 10 -dump
\end{verbatim}


\section{Command-Line Tool}

The basic syntax for the command-line execution is as follows:
\begin{grammar}
<command> ::= `hysia' <options> <filename>
\end{grammar}
%
The \texttt{hysia} command accepts the following options:
\begin{description}
\item[\texttt{-h}, \texttt{-help}, or \texttt{--help}] Displays a summary of the options accepted by the command.
\item[\texttt{-n}] Specifies the number steps to simulate (default is $\infty$).
\item[\texttt{-t}] Specifies the max simulation time (default is $\infty$).
\item[\texttt{-a}] Decides the simulation length automatically from the STL property.
\item[\texttt{-g}] Sets the debug flag.
\item[\texttt{-dump}] Activates dumping plot to the file ``pped.dat.''
\item[\texttt{-cm_thres}] Sets the threshold for character matrix selection.
\end{description}

The option \texttt{-cm_thres} specifies the character matrix $B$ to be used in the parallelotope method (see the corresponding publication for the detail). It is selected as follows:
\begin{itemize}
	\item $-1$: $B := (\mathrm{mid}\J) A$.
	\item $0$: $B := I$ (i.e., identity matrix).
	\item $1$: $B := \mathrm{orthogonalize}((\mathrm{mid}\J) A)$.
	\item $n > 1$ (default): 
		\[
			B := \begin{cases}
				(\mathrm{mid}\J) A & \text{if $\kappa((\mathrm{mid}\J) A) < n$} \\
				\mathrm{orthogonalize}((\mathrm{mid}\J) A) & \text{otherwise.}
			\end{cases}
		\]
\end{itemize}


\section{Solving Parameters}

\begin{description}
\item[\texttt{order}] Order of Taylor expansion.
\item[\texttt{t_max}] Max time horizon assumed in the simulation of each step.
\item[\texttt{h_min}] Min time CAPD integration can take.
\item[\texttt{epsilon}] Specifies the precision of the event detection.
\item[\texttt{dump_interval}] Sets the precision of the dumped flowpipe data.
\item[\texttt{delta}] Parameter for the box inflation process.
\item[\texttt{tau}] Parameter for the box inflation process.
\item[\texttt{cm_thres}] Parameter for the character matrix selection.
\end{description}


\section{Specification Language}

This section describes the grammar of the specification language of HySIA.
A specification consists of the definition of a hybrid automaton, an STL formula, and solving parameter configurations.

\subsection{Lexical conventions}

The lexical class of digits, letters, and identifiers is the following:
\begin{grammar}
<digit> ::= [`0'--`9']

<letter> ::= [`a'--`z' `A'--`Z']

<id> ::= <letter> (<digit> | <letter> | `_')*
\end{grammar}

The syntax of various numeral expressions is as follows:
\begin{grammar}
<integer> ::= <digit>+

<float> ::= 
%<digit>+ ~~|~~ <digit>+ `.' <digit>*
<digit>+ ~ (`.' <digit>*)? ~ ( (`e'|`E') (`+'|`-')? <digit>+ )?
%\alt <digit>+ `.' <digit>* (`e'|`E')  <digit>+
%\alt <digit>+ (`e'|`E') <digit>+
%\alt <digit>+ (`e'|`E') (`+'|`-') <digit>+

<float-pn> ::= <float> ~~|~~ `-' <float>

<interval> ::= <interval-noun> ~|~ <float-pn>

<interval-noun> ::= `(' <float-pn> `,' <float-pn> `)'

<interval-list> ::= <interval> <interval-list-rest>
\alt `(' <interval> <interval-list-rest> `)'

<interval-list-rest> ::= `,' <interval> <interval-list-rest> ~~|~~ <empty>
\end{grammar}


\noindent
\textbf{Comments.}
Comments are either enclosed between \texttt{(*} and \texttt{*)} (can be nested) or prefixed with \texttt{\#}.


\subsection{Toplevel syntax}

The syntax for the toplevel of specifications is the following:
\begin{grammar}
<specification> ::= <statement>+ <property>? <solver-param>*

<statement> ::= 
`let' <id> `=' <interval>
\alt `let' <id> `=' `R' <float>
\alt `var' <var-list>
\alt `init' <expr-list>
\alt `at' <id> <flow> <invariant>? <edge>* `end'

<property> ::= `prop' <mitl-formula>

<solver-param> ::= `param' <id> `=' <float-pn>


<flow> ::= `wait' <expr-list>

<invariant> ::= `inv' <expr-list>

<edge> ::= (`when' | `once') `(' <expr> `,' <expr-list> `)' `goto' <id> `then' <expr-list>
\end{grammar}


\subsection{Expressions}

The operators and function application in expressions have the priorities and associativities as shown in the table below (from lowest to greatest priority):
\begin{table}[ht]
	\centering
	%\caption{\label{t:operators} Terms} 
	%\small
    \begin{tabular}{|l|l|} \hline
		construct & associativity \\
		\hline
		`\texttt{+}', `\texttt{-}' & left \\
		`\texttt{*}', `\texttt{/}' & left \\
		function application & left \\
		`\texttt{\^}', `\texttt{-}' (unary) & --- \\
		\hline
	\end{tabular}
\end{table}

The syntax for expressions is the following:
\begin{grammar}
<expr> ::= <expr> `+' <expr> ~~|~~ <expr> `-' <expr>
\alt <expr> `*' <expr> ~~|~~ <expr> `/' <expr>
\alt `-' <expr> ~~|~~ <expr> `^' <integer>
\alt <function> <expr>
\alt <id> ~~|~~ <interval>
\alt `(' <expr> `)' 

<function> ::= `sqrt' | `exp' | `log' | `sin' | `cos' | `atan' | `asin' | `acos'

<expr-list> ::= <expr> <expr-list-rest> ~~|~~ `(' <expr> <expr-list-rest> `)'

<expr-list-rest> ::= `,' <expr> <expr-list-rest> ~|~ <empty>
\end{grammar}


\subsection{STL formulae}

The operators in formulae have the following priorities and associativities:
\begin{table}[ht]
	\centering
	%\caption{\label{t:operators} Terms} 
	%\small
    \begin{tabular}{|l|l|} \hline
		construct & associativity \\
		\hline
		`\texttt{->}' & right \\
		`\texttt{|}' & right \\
		`\texttt{\&}' & right \\
		`\texttt{U} [$\cdot$,$\cdot$]' & right \\
		`\texttt{G} [$\cdot$,$\cdot$]', `\texttt{F} [$\cdot$,$\cdot$]' & --- \\
		`\texttt{!}' & --- \\
		\hline
	\end{tabular}
\end{table}

The syntax for formulae is the following:
\begin{grammar}
<mitl-formula> ::= `true' ~~|~~ `false'
\alt <id> ~~|~~ <expr>
\alt `!' <mitl-formula>
\alt <mitl-formula> (`&'|`|'|`->') <mitl-formula>
\alt (`F'|`G') <noun-interval> <mitl-formula>
\alt <mitl-formula> `U' <noun-interval> <mitl-formula>
\alt `(' <mitl-formula> `)'
\end{grammar}

\end{document}
